% Chapter 1

\chapter{Introduction} % Main chapter title

\label{Introduction} % For referencing the chapter elsewhere, use \ref{Chapter1} 

%----------------------------------------------------------------------------------------

% Define some commands to keep the formatting separated from the content 
\newcommand{\keyword}[1]{\textbf{#1}}
\newcommand{\tabhead}[1]{\textbf{#1}}
\newcommand{\code}[1]{\texttt{#1}}
\newcommand{\file}[1]{\texttt{\bfseries#1}}
\newcommand{\option}[1]{\texttt{\itshape#1}}

%----------------------------------------------------------------------------------------

\section{Motivation}
Digitalisation places a special emphasis on companies' ability to react to changes in circumstances, such as changes in the law or changing customer needs. In the near future, business models that are currently successful may not necessarily work in rapidly changing environments \cite{BBA}.

Typically, companies which are not digitally native rely on third-party software. To adapt their products or services to the changing demands, they have a tedious process. If they are too slow, these companies may miss out on key business opportunities, while if they are too fast, they may fail to deliver quality software \cite{BBA}.

Model-driven engineering (MDE) can come here in place in order to reduce time from requirements to implementation. MDE is defined as a process in which domain experts such as product managers capture requirements through models. In theory, these models can be automatically transformed in executable code. However, the reality is that MDE is rarely used in general software development today, but rather in some niches. It is assumed that the success of MDE in these niches depends on the use of domain-specific languages that allow domain experts to express requirements in a formal language that is natural to both them and software developers, as in contrast to other domains that lack a modelling language as common basis \cite{BBA}.

Microsoft Office documents could serve as such a common basis, as they have a high acceptance in the working world. Also, Office documents often contain valuable domain data and business logic which is later manually converted into code by software developers. When applying MDE, domain experts could thus easily reflect changes in the environment in a Office document from which an updated version of the software can be automatically be generated.  In this way, companies are enabling themselves to quickly respond to changing business requirements by integrating business artefacts from domain-specific languages (e.g Microsoft Excel.) into their software development life cycle \cite{BBA}.

There are existing approaches how Microsoft Excel could be used as a micro service with the help of a python package called "formulas" \cite{BBA}. With the use of code generation and model interpretation it was made possible to integrate an Excel document into an existing application as first-class modeling artefact.
However, first tests have shown that it is a rather slow solution performancewise. Also, not all functions types are are supported which can be performed by Microsoft Excel. Thus, it is necessary to find other solutions which are able to overcome these limitations \cite{BBA}. 





%----------------------------------------------------------------------------------------

\section{Motivation Example}

The validation of the different approaches is based on an example from industry that uses an Excel document to calculate the required spare parts for electrical devices. The calculations defined in the Excel document are to be made available as a micro-service. 
The spare parts listed in the Excel file generate costs for their storage and preservation. It is therefore a natural goal to have as few of them as possible. On the other hand, the largest possible stock should be ensured so that in case of failures, they can be repaired immediately, thus reducing the economic impact. According to various input data, the Excel document calculates the optimal stock of spare parts so that the total costs are minimised.

...

\section{Objective and Structure}

%hier evtl noch besser an Motivation Teil anknüpfen
The aim of this VT is to create a web application that can interact with an Excel document that is available as a micro service. The user of the web application should be able to call up the micro service by entering the inputs in an input mask. When the service is called, it should write the parameters into predefined fields of the Excel file, calculate the formulas in the file and return predefined cell values as a result and display them in the UI of the web application. 
The interaction with the Excel document is to be based on the "model at runtime" principle. Microsoft Graph is to be used for interaction with the model (i.e. the Excel document). Microsoft Graph provides a REST API for Excel documents stored in the cloud to enable their integration into web applications. The performance and limitations of the technology will be evaluated and compared  to existing solution which make us of the python package "formulas" \cite{BBA}.

Furthermore, it should be possible to generate the UI of the web application automatically from the Excel document (e.g. from predefined input and output fields of the Excel document). Alternatively, an approach should be developed that allows the Excel document to be integrated into an existing web application. In this case, inputs for calculations should not come from an input mask, but should be obtained from an integrated database that has the required data.

%To be able to do research on the topic, Chapter 2 summarises the theoretical foundation for MDSE in general and regarding to the two principles of models at compile time and models at runtime. After the theoretical foundation, chapter 3 gives an insight into the specific case of using Microsoft Office documents in the context of MDSE. In Chapter 5, possible technologies for using Microsoft Office documents are evaluated in terms of compile time models and runtime models. The technologies are evaluated regarding the aspects defined in chapter 4. Finally, in chapter 6, a summary and a hint for future work are given.


\section{Methodology}

tbd

%To answer the research question appropriately, a literature review is conducted to gain an understanding of MDSE and its theoretical foundation. The basic understanding of the topic gained subsequently enables to do research on using Microsoft Office in the context of MDSE. Furthermore, possible technologies to interact with documents are evaluated according to the following aspects:
%•	Dependence on a specific Office version
%•	License costs
%•	Performance
%•	Ease of use
%•	Completeness of generated code
%•	Testability
%For each of the technologies found, a description of how the technology works and helps to implement the desired principle is given. Also, the tools and libraries needed for the performed evaluation are documented. Where possible, the feasibility of the tech-nology for the desired principles is investigated with a short example programme. Final-ly, limitations of the chosen technologies are shown.


